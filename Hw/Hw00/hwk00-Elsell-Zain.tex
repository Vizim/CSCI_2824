 \documentclass[9pt]{article}
 \usepackage{amsmath}
 \usepackage{amssymb}
 \usepackage{graphicx}    % needed for including graphics e.g. EPS, PS \usepackage{tikz}
 \usepackage{tikz}
 \usepackage{relsize}
 \usetikzlibrary{patterns,decorations.pathreplacing,shapes,arrows}
 \usepackage{algorithm2e}
 \topmargin -2.5cm        % read Lamport p.163
 \oddsidemargin -0.04cm   % read Lamport p.163
 \evensidemargin -0.04cm  % same as oddsidemargin but for left-hand pages
 \textwidth 16.59cm
 \textheight 25.94cm
% \pagestyle{empty}        % Uncomment if don't want page numbers
 \pagenumbering{gobble}
 \parskip 7.2pt           % sets spacing between paragraphs
 %\renewcommand{\baselinestretch}{1.5} 	% Uncomment for 1.5 spacing between lines
 \parindent 0pt		  % sets leading space for paragraphs

% No date in header
\date{}

\newcommand{\lp}{\left(}
\newcommand{\rp}{\right)}
\newcommand{\lb}{\left[}
\newcommand{\rb}{\right]}
\newcommand{\ls}{\left\{}
\newcommand{\rs}{\right\}}
\newcommand{\lbar}{\left|}
\newcommand{\rbar}{\right|}
\newcommand{\ld}{\left.}
\newcommand{\rd}{\right.}

\newcommand{\myexists}{\exists \hspace{.3mm}}

\newcommand{\hs}{\hspace{.75mm}}
\newcommand{\bs}{\hspace{-.75mm}}
\newcommand{\nin}{\noindent}

\newcommand{\fx}{f\bs\left( x \right)}
\newcommand{\gx}{g\bs\left( x \right)}
\newcommand{\qx}{q\bs\left( x \right)}

\newcommand{\nn}{\nonumber}

\newcommand{\vfive}{\vspace{5mm}}
\newcommand{\vthree}{\vspace{3mm}}

\newcommand{\fof}[1]{f\lp #1\rp}
\newcommand{\gof}[1]{g\lp #1\rp}
\newcommand{\qof}[1]{q\lp #1\rp}

\newcommand{\myp}[1]{\left( #1 \right)}
\newcommand{\myb}[1]{\left[ #1 \right]}
\newcommand{\mys}[1]{\left\{ #1 \right\}}
\newcommand{\myab}[1]{\left| #1 \right|}

\newcommand{\myj}{_j}
\newcommand{\myjp}{_{j+1}}
\newcommand{\myjm}{_{j-1}}

\newcommand{\f}[1]{f\hspace{-1mm}\left( #1 \right)}
\newcommand{\fp}[1]{f'\hspace{-1mm}\left( #1 \right)}
\newcommand{\g}[1]{g\hspace{-1mm}\left( #1 \right)}
\newcommand{\gp}[1]{g'\hspace{-1mm}\left( #1 \right)}
\newcommand{\q}[1]{q\hspace{-1mm}\left( #1 \right)}
\newcommand{\qp}[1]{q'\hspace{-1mm}\left( #1 \right)}
\newcommand{\Px}[1]{P\hspace{-1mm}\left( x_{#1} \right)}
\newcommand{\Qx}[1]{Q\hspace{-1mm}\left( x_{#1} \right)}

\newcommand{\tten}[1]{\times 10^{#1}}

\newcommand{\aij}[1]{a_{#1}}
\newcommand{\bij}[1]{b_{#1}}
\newcommand{\rij}[1]{r_{#1}}

\newcommand{\R}[1]{\mathbb{R}^{#1}}

\newcommand{\ith}{i^{\textrm{th}}}
\newcommand{\jth}{i^{\textrm{th}}}
\newcommand{\kth}{i^{\textrm{th}}}

\newcommand{\inv}[1]{{#1}^{-1}}

\newcommand{\bx}{\mathbf{x}}
\newcommand{\bv}{\mathbf{v}}
\newcommand{\bw}{\mathbf{w}}
\newcommand{\by}{\mathbf{y}}
\newcommand{\bb}{\mathbf{b}}
\newcommand{\be}{\mathbf{e}}
\newcommand{\br}{\mathbf{r}}
\newcommand{\xhat}{\hat{\mathbf{x}}}

\newcommand{\beq}{\begin{eqnarray}}
\newcommand{\eeq}{\end{eqnarray}}

\newcommand{\ben}{\begin{enumerate}}
\newcommand{\een}{\end{enumerate}}

\newcommand{\bsq}{\mathsmaller{\blacksquare}}

\newcommand{\iter}[1]{^{\myp{#1}}}

% matrix macro
\newcommand{\mymat}[1]{
\left[
\begin{array}{rrrrrrrrrrrrrrrrrrrrrrrrrrrrrrrrrrrrrrr}
#1
\end{array}
\right]
}

\newcommand{\smallaug}[1]{
\left[
\begin{array}{rr|r}
#1
\end{array}
\right]
}


% Actual document starts here
% ======================================================================================
\begin{document}

\begin{minipage}{0.85\textwidth}
\nin {\bf CSCI 2824 -- Spring 2022 }
\end{minipage}\hfill
\begin{minipage}{0.15\textwidth}
\nin \hfill {\bf Homework 0}
\end{minipage}


%\vfive

\begin{center}\textbf{Read these instructions!}\end{center}

\nin This assignment is \textbf{due on Friday, January 14 to Gradescope by 6 PM.}  You are expected to write up your solutions neatly, with {\bf full explanations and justifications} when necessary.  Remember that you are encouraged to discuss problems with your classmates, but you must work and write your solutions on your own. {\bf Important}: On the {\bf FRONT} of your assignment clearly write your full name and the lecture section you belong to.  You may upload your assignment as a pdf or as images of your work. Make sure that your images/scans are clear or you will lose points/possibly be given a 0.
Also, if you do not match your questions on Gradescope, then no credit will be given.

\ben

% ======================================================================================
% Question 1 here
% ======================================================================================

\item According to the syllabus: \\
\\
\ben
\item[a] ) What is Piazza used for?\\
\\
\item[b] ) What is Gradescope used for?\\
\\
\item[c])  What is Zoom used for?\\
\\
\item[d] ) What is Canvas used for?\\
\\
\een
\vspace{3mm}
% ======================================================================================
% Question 2 here
% ======================================================================================

% \vspace{2mm}
\item Due dates:
\ben
\item[a]) What day and time is homework due every week?\\
\\
\item[b]) What day and time are quizzes open every week (except exam week)?\\
\een
% ======================================================================================
% Question 3 here
% ======================================================================================

\item The following questions are about determining your grade:

\ben
\item What is the total amount of points available this semester and how can a student get 54/50 on homework?\\
\\
\item If you obtain 900 points by the end of the semester, then what would your grade percentage be?\\
\\
\een

\vspace{3mm}

% ======================================================================================
% Question 4 here
% ======================================================================================

\item For this class, what programming language will we be coding in?\\
\\

\vspace{3mm}

% ======================================================================================
% Question 5 here
% ======================================================================================

\item On $\textbf{average}$, how many hours of out-of-class study time will generate a 'B' or 'C' grade in this course?\\
\\

\vspace{3mm}

% ======================================================================================
% Question 6 here
% ======================================================================================

\item Suppose at the end of the semester your grade is borderline between 2 different grades. Asking to be 'pushed up' will not get your grade changed, but what WILL get your grade boosted at that point?\\
\\

\vspace{5mm}


\een

\end{document}
